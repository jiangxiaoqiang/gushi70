\documentclass[a4paper,zihao=-4,oneside,landscape,UTF8]{ctexart}

%页面旋转
\usepackage{everypage}
\AddEverypageHook{\special{pdf: put @thispage <</Rotate 90>>}}

%% font文字旋转
\defaultCJKfontfeatures{RawFeature={vertical:+vert}}

%% 默认字体为源雲明體
\setCJKmainfont[BoldFont=源雲明體 TTF SemiBold,ItalicFont=思源宋体 Light]{源雲明體 TTF Medium}

% 页码编号为汉字
\renewcommand{\thepage}{\Chinese{page}}

\usepackage{geometry}
\newgeometry{top=50pt, bottom=100pt, left=100pt, right=80pt,headsep=0pt}

\usepackage{titlesec}
\titleformat{\section}[hang]{\large\bfseries}{}{0pt}{}
\titlespacing{\section}{4em}{0pt}{0pt}


\ctexset{
	today = big,
	punct = quanjiao, 
	autoindent = 0pt,	
}

%去掉页眉
\usepackage{fancyhdr}
\renewcommand{\headrulewidth}{0.0pt}%
\fancypagestyle{plain}{% Redefine plain pages tyle  
  % Clear header/footer
  \fancyhf{}
  \renewcommand{\headrulewidth}{0.0pt}%
  \fancyfoot[L]{\thepage}
}
\pagestyle{plain}

%%% 命令设置
\usepackage{xcolor}
\definecolor{gray}{gray}{0.3}
\newcommand{\zhushi}[1]{\scriptsize{\textit{\textcolor{gray}{#1}}}\normalsize}

\definecolor{gray1}{gray}{0.2}
\color{gray1}


\title{\textbf{老子}\textit{ 注 \hspace{5em}河上公}\hfil}
\author{}
\date{\normalsize\today 版}

\begin{document}

\ziju{0.1}
\large

\maketitle


\section{体道第一}


道可道,
\zhushi{谓经术政教之道也。}
非常道。
\zhushi{非自然生长之道也。常道当以无为养神,无事安民,含光藏晖,灭迹匿端,不可称道。}
名可名,
\zhushi{谓富贵尊荣,高世之名也。}
非常名。
\zhushi{非自然常在之名也。常名当如婴儿之未言,鸡子之未分,明珠在蚌中,美玉处石间,内虽昭昭,外如愚顽。}
无名,天地之始。
\zhushi{无名者谓道,道无形,故不可名也。始者道本也,吐气布化,出于虚无,为天地本始也。}
有名,万物之母。
\zhushi{有名谓天地。天地有形位、有阴阳、有柔刚,是其有名也。万物母者,天地含气生万物,长大成熟,如母之养子也。}
故常无欲,以观其妙;
\zhushi{妙,要也。人常能无欲,则可以观道之要,要谓一也。一出布名道,赞叙明是非。}
常有欲,以观其徼。
\zhushi{徼,归也。常有欲之人,可以观世俗之所归趣也。}
此两者,同出而异名,
\zhushi{两者,谓有欲无欲也。同出者,同出人心也。而异名者,所名各异也。名无欲者长存,名有欲者亡身也。}
同谓之玄,
\zhushi{玄,天也。言有欲之人与无欲之人,同受气于天。}
玄之又玄,
\zhushi{天中复有天也。禀气有厚薄,得中和滋液,则生贤圣,得错乱污辱,则生贪淫也。}
众妙之门。
\zhushi{能知天中复有天,禀气有厚薄,除情去欲守中和,是谓知道要之门户也。}


\section{养身第二}

天下皆知美之为美,
\zhushi{自扬己美,使彰显也。}
斯恶已;
\zhushi{有危亡也。}
皆知善之为善,
\zhushi{有功名也。}
斯不善已。
\zhushi{人所争也。}
故有无相生,
\zhushi{见有而为无也。}
难易相成,
\zhushi{见难而为易也。}
长短相较,
\zhushi{见短而为长也。}
高下相倾,
\zhushi{见高而为下也。}
音声相和,
\zhushi{上唱下必和也。}
前后相随。
\zhushi{上行下必随也。}
是以圣人处无为之事,
\zhushi{以道治也。}
行不言之教,
\zhushi{以身师导之也。}
万物作焉
\zhushi{各自动也。}
而不辞,
\zhushi{不辞谢而逆止。}
生而不有,
\zhushi{元气生万物而不有}
为而不恃,
\zhushi{道所施为,不恃望其报也。}
功成而弗居。
\zhushi{功成事就,退避不居其位。}
夫惟弗居,
\zhushi{夫惟功成不居其位。}
是以不去。
\zhushi{福德常在,不去其身也。此言不行不可随,不言不可知疾。上六句有高下长短,君开一源,下生百端,百端之变,无不动乱。}


\section{安民第三}

不尚贤,
\zhushi{贤谓世俗之贤,辩口明文,离道行权,去质为文也。不尚者,不贵之以禄,不贵之以官。}
使民不争。
\zhushi{不争功名,返自然也。}
不贵难得之货,
\zhushi{言人君不御好珍宝,黄金弃于山,珠玉捐于渊。}
使民不为盗。
\zhushi{上化清静,下无贪人。}
不见可欲,
\zhushi{放郑声,远美人。}
使心不乱。
\zhushi{不邪淫,不惑乱也。}
是以圣人之治,
\zhushi{说圣人治国与治身同也。}
虚其心,
\zhushi{除嗜欲,去乱烦。}
实其腹,
\zhushi{怀道抱一守,五神也。}
弱其志,
\zhushi{和柔谦让,不处权也。}
强其骨。
\zhushi{爱精重施,髓满骨坚。强,其良反}
常使民无知无欲。
\zhushi{返朴守淳。}
使夫智者不敢为也。
\zhushi{思虑深,不轻言。夫音符,知音智}
为无为,
\zhushi{不造作,动因循。}
则无不治。
\zhushi{德化厚,百姓安。}


\section{无源第四}

道冲而用之
\zhushi{冲,中也。道匿名藏誉,其用在中。冲,直隆反}
或不盈,
\zhushi{或,常也。道常谦虚不盈满。}
渊乎似万物之宗。
\zhushi{道渊深不可知,似为万物之宗祖。}
挫其锐,
\zhushi{锐,进也。人欲锐精进取功名,当挫止之,法道不自见也。}
解其纷,
\zhushi{纷,结恨也。当念道无为以解释。}
和其光,
\zhushi{言虽有独见之明,当知暗昧,不当以擢乱人也。}
同其尘。
\zhushi{当与众庶同垢尘,不当自别殊。}
湛兮似若存。
\zhushi{言当湛然安静,故能长存不亡。}
吾不知谁之子,
\zhushi{老子言:我不知,道所从生。}
象帝之先。
\zhushi{道自在天帝之前,此言道乃先天地之生也。至今在者,以能安静湛然,不劳烦欲使人修身法道。}


\section{虚用第五}

天地不仁,
\zhushi{天施地化,不以仁恩,任自然也。}
以万物为刍狗。
\zhushi{天地生万物,人最为贵,天地视之如刍草狗畜,不贵望其报也。}
圣人不仁,
\zhushi{圣人爱养万民,不以仁恩,法天地行自然。}
以百姓为刍狗。
\zhushi{圣人视百姓如刍草狗畜,不贵望其礼意。}
天地之间,
\zhushi{天地之间空虚,和气流行,故万物自生。人能除情欲,节滋味,清五脏,则神明居之也。}
其犹橐龠乎。
\zhushi{橐龠中空虚,人能有声气。}
虚而不屈,动而愈出。
\zhushi{言空虚无有屈竭时,动摇之,益出声气也。}
多言数穷,
\zhushi{多事害神,多言害身,口开舌举,必有祸患。}
不如守中。
\zhushi{不如守德于中,育养精神,爱气希言。}


\section{成象第六}

谷神不死,
\zhushi{谷,养也。人能养神则不死也。神,谓五脏之神也。肝藏魂,肺藏魄,心藏神,肾藏精,脾藏志,五藏尽伤,则五神去矣。}
是谓玄牝。
\zhushi{言不死之有,在于玄牝。玄,天也,于人为鼻。牝,地也,于人为口。天食人以五气,从鼻入藏于心。五气轻微,为精、神、聪、明、音声五性。其鬼曰魂,魂者雄也,主出入于人鼻,与天通,故鼻为玄也。地食人以五味,从口入藏于胃。五味浊辱,为形、骸、骨、肉、血、脉六情。其鬼曰魄,魄者雌也,主出入于人口,与地通,故口为牝也。}
玄牝之门,是谓天地根。
\zhushi{根,元也。言鼻口之门,是乃通天地之元气所从往来也。}
绵绵若存,
\zhushi{鼻口呼噏喘息,当绵绵微妙,若可存,复若无有。}
用之不勤。
\zhushi{用气当宽舒,不当急疾勤劳也。}


\section{韬光第七}

天长地久,
\zhushi{说天地长生久寿,以喻教人也。 }
天地所以能长且久者,以其不自生,
\zhushi{天地所以独长且久者,以其安静,施不求报,不如人居处,汲汲求自饶之利,夺人以自与也。}
故能长生。
\zhushi{以其不求生,故能长生不终也。}
是以圣人后其身,
\zhushi{先人而后己也。}
而身先,
\zhushi{天下敬之,先以为长。}
外其身,
\zhushi{薄己而厚人也。}
而身存。
\zhushi{百姓爱之如父母,神明佑之若赤子,故身常存。}
非以其无私邪。
\zhushi{圣人为人所爱,神明所佑,非以其公正无私所致乎。}
故能成其私。
\zhushi{人以为私者,欲以厚己也。圣人无私而己自厚,故能成其私也。}


\section{易性第八}

上善若水。
\zhushi{上善之人,如水之性。}
水善利万物而不争,
\zhushi{水在天为雾露,在地为源泉也。}
处众人之所恶,
\zhushi{众人恶卑湿垢浊,水独静流居之也。}
故几于道。
\zhushi{水性几于道同。}
居善地,
\zhushi{水性善喜于地,草木之上即流而下,有似于牝动而下人也。}
心善渊,
\zhushi{水深空虚,渊深清明。}
与善仁,
\zhushi{万物得水以生。与,虚不与盈也。}
言善信,
\zhushi{水内影照形,不失其情也。}
正善治,
\zhushi{无有不洗,清且平也。}
事善能,
\zhushi{能方能圆,曲直随形。}
动善时。
\zhushi{夏散冬凝,应期而动,不失天时。}
夫唯不争,
\zhushi{壅之则止,决之则流,听从人也。}
故无尤。
\zhushi{水性如是,故天下无有怨尤水者也。}


\section{运夷第九}

持而盈之,不如其已。
\zhushi{盈,满也。已,止也。持满必倾,不如止也。}
揣而梲之,不可长保。
\zhushi{揣,治也。先揣之,后必弃捐。}
金玉满堂,莫之能守。
\zhushi{嗜欲伤神,财多累身。}
富贵而骄,自遗其咎。
\zhushi{夫富当赈贫,贵当怜贱,而反骄恣,必被祸患也。}
功成、名遂、身退,天之道。
\zhushi{言人所为,功成事立,名迹称遂,不退身避位,则遇于害,此乃天之常道也。譬如日中则移,月满则亏,物盛则衰,乐极则哀。}


\section{能为第十}

载营魄,
\zhushi{营魄,魂魄也。人载魂魄之上得以生,当爱养之。喜怒亡魂,卒惊伤魄。魂在肝,魄在肺。美酒甘肴,腐人肝肺。故魂静志道不乱,魄安得寿延年也。}
抱一,能无离乎,
\zhushi{言人能抱一,使不离于身,则长存。一者,道始所生,太言道行德,玄冥不可得见,欲使人如道也。}


\section{无用第十一}

三十辐共一毂,
\zhushi{古者车三十辐,法月数也。共一毂者,毂中有孔,故众辐共凑之。治身者当除情去欲,使五藏空虚,神乃归之。治国者寡能,摠众弱共使强也。}
当其无,有车之用。
\zhushi{无,谓空虚。毂中空虚,轮得转行,舆中空虚,人得载其上也。}
埏埴以为器,
\zhushi{埏,和也。埴,土也。和土以为饮食之器。}
当其无,有器之用。
\zhushi{器中空虚,故得有所盛受。}
凿户牖以为室,
\zhushi{谓作屋室。}
当其无有室之用。
\zhushi{言户牖空虚,人得以出入观视;室中空虚,人得以居处,是其用。}
故有之以为利,
\zhushi{利,物也,利于形用。器中有物,室中有人,恐其屋破坏,腹中有神,畏其形亡也。}
无之以为用。
\zhushi{言虚空者乃可用盛受万物,故曰虚无能制有形。道者空也。}


\section{检欲第十二}

五色令人目盲;
\zhushi{贪淫好色,则伤精失明也。}
五音令人耳聋;
\zhushi{好听五音,则和气去心,不能听无声之声。}
五味令人口爽;
\zhushi{爽,亡也。人嗜于五味于口,则口亡,言失于道也。}
驰骋畋猎,令人心发狂,
\zhushi{人精神好安静,驰骋呼吸,精神散亡,故发狂也。}
难得之货,令人行妨。
\zhushi{妨,伤也。难得之货,谓金银珠玉,心贪意欲,不知餍足,则行伤身辱也。}
是以圣人为腹,
\zhushi{守五性,去六情,节志气,养神明。}
不为目,
\zhushi{目不妄视,妄视泄精于外。}
故去彼取此。
\zhushi{去彼目之妄视,取此腹之养性。}


\section{厌耻第十三}

宠辱若惊,
\zhushi{身宠亦惊,身辱亦惊。}
贵大患若身。
\zhushi{贵,畏也。若,至也。谓大患至身,故皆惊。}
何谓宠辱。
\zhushi{问何谓宠,何谓辱。宠者尊荣,辱者耻辱。及身还自问者,以晓人也。}
辱为下,
\zhushi{辱为下贱。}
得之若惊,
\zhushi{得宠荣惊者,处高位如临深危也。贵不敢骄,富不敢奢。}
失之若惊,
\zhushi{失者,失宠处辱也。惊者,恐祸重来也。}
是谓宠辱若惊。
\zhushi{解上得之若惊,失之若惊。}
何谓贵大患若身。
\zhushi{复还自问:何故畏大患至身。}
吾所以有大患者,为吾有身。
\zhushi{吾所以有大患者,为吾有身。有身忧者,勤劳念其饥寒,触情从欲,则遇祸患也。}
及吾无身,吾何有患。
\zhushi{使吾无有身体,得道自然,轻举升云,出入无间,与道通神,当有何患。}
故贵以身为天下者,则可寄天下,
\zhushi{言人君贵其身而贱人,欲为天下主者,则可寄立,不可以久也。}
爱以身为天下,若可托天下。
\zhushi{言人君能爱其身,非为己也,乃欲为万民之父母。以此得为天下主者,乃可以托其身于万民之上,长无咎也。}


\section{赞玄第十四}


视之不见名曰夷,
\zhushi{无色曰夷。言一无采色,不可得视而见之。}
听之不见名曰希,
\zhushi{无声曰希。言一无音声,不可得听而闻之。}
搏之不得名曰微。
\zhushi{无形曰微。言一无形体,不可抟持而得之。}
此三者不可致诘,
\zhushi{三者,谓夷、希、微也。不可致诘者,夫无色、无声、无形,口不能言,书不能传,当受之以静,求之以神,不可问诘而得之也。}
故混而为一。
\zhushi{混,合也。故合于三名之为一。}
其上不皦,
\zhushi{言一在天上,不皦。皦,光明。}
其下不昧。
\zhushi{言一在天下,不昧。昧,有所暗冥。}
绳绳不可名,
\zhushi{绳绳者,动行无穷级也。不可名者,非一色也,不可以青黄白黑别,非一声也,不可以宫商角徵羽听,非一形也,不可以长短大小度之也。}
复归于无物。
\zhushi{物,质也。复当归之于无质。}
是谓无状之状,
\zhushi{言一无形状,而能为万物作形状也。}
无物之象,
\zhushi{一无物质,而为万物设形象也。}
是谓惚恍。
\zhushi{一忽忽恍恍者,若存若亡,不可见之也。}
迎之不见其首,
\zhushi{一无端末,不可预待也。除情去欲,一自归之也。}
随之不见其后,
\zhushi{言一无影迹,不可得而看。}
执古之道,以御今之有,
\zhushi{圣人执守古道,生一以御物,知今当有一也。}
能知古始,是谓道纪。
\zhushi{人能知上古本始有一,是谓知道纲纪也。}


\section{显德第十五}

古之善为士者,
\zhushi{谓得道之君也。}
微妙玄通,
\zhushi{玄,天也。言其志节玄妙,精与天通也。}
深不可识。
\zhushi{道德深远,不可识知,内视若盲,反听若聋,莫知所长。}
夫唯不可识,故强为之容。
\zhushi{谓下句也。}
与兮若冬涉川;
\zhushi{举事辄加重慎与。与兮若冬涉川,心难之也。}
犹兮若畏四邻;
\zhushi{其进退犹犹如拘制,若人犯法,畏四邻知之也。}
俨兮其若容;
\zhushi{如客畏主人,俨然无所造作也。}
涣兮若冰之将释,
\zhushi{涣者,解散。释者,消亡。除情去欲,日以空虚。}
敦兮其若朴,
\zhushi{敦者,质厚。朴者,形未分。内守精神,外无文采也。}
旷兮其若谷;
\zhushi{旷者,宽大。谷者,空虚。不有德功名,无所不包也。}
浑兮其若浊。
\zhushi{浑者,守本真,浊者,不照然。与众合同,不自专也。}
孰能浊以静之,徐清。
\zhushi{孰,谁也。谁能知水之浊止而静之,徐徐自清也。}
孰能安以久动之,徐生。
\zhushi{谁能安静以久,徐徐以长生也。}
保此道者,不欲盈。
\zhushi{保此徐生之道,不欲奢泰盈溢。}
夫惟不盈,故能蔽不新成。
\zhushi{夫为不盈满之人,能守蔽不为新成。蔽者,匿光荣也。新成者,贵功名。}


\section{归根第十六}

致虚极,
\zhushi{得道之人,捐情去欲,五内清静,至于虚极。}
守静笃,
\zhushi{守清静,行笃厚。}
万物并作,
\zhushi{作,生也。万物并生也。}
吾以观复。
\zhushi{言吾以观见万物无不皆归其本也。人当念重其本也。}
夫物芸芸,
\zhushi{芸芸者,华叶盛也。}
各复归其根,
\zhushi{言万物无不枯落,各复反其根而更生也。}
归根曰静,
\zhushi{静谓根也。根安静柔弱,谦卑处下,故不复死也。}
是谓复命。
\zhushi{言安静者是为复还性命,使不死也。}
复命曰常。
\zhushi{复命使不死,乃道之所常行也。}
知常曰明;
\zhushi{能知道之所常行,则为明。}
不知常,妄作凶。
\zhushi{不知道之所常行,妄作巧诈,则失神明,故凶也。}
知常容,
\zhushi{能知道之所常行,去情忘欲,无所不包容也。}
容乃公,
\zhushi{无所不包容,则公正无私,众邪莫当。}
公乃王,
\zhushi{公正无私,可以为天下王。治身正则形一,神明千万,共凑其躬也。}
王乃天,
\zhushi{能王,德合神明,乃与天通。}
天乃道,
\zhushi{德与天通,则与道合同也。}
道乃久。
\zhushi{与道合同,乃能长久。}
没身不殆。
\zhushi{能公能王,通天合道,四者纯备,道德弘远,无殃无咎,乃与天地俱没,不危殆也。}


\section{淳风第十七}

太上,下知有之。
\zhushi{太上,谓太古无名之君。下知有之者,下知上有君,而不臣事,质朴也。}
其次,亲之誉之。
\zhushi{其德可见,恩惠可称,故亲爱而誉之。}
其次畏之。
\zhushi{设刑法以治之。}
其次侮之。
\zhushi{禁多令烦,不可归诚,故欺侮之。}
信不足焉,﹝有不信焉﹞。
\zhushi{君信不足于下,下则应之以不信,而欺其君也。}
犹兮其贵言。
\zhushi{说太上之君,举事犹,贵重于言,恐离道失自然也。}
功成事遂,
\zhushi{谓天下太平也。}
百姓皆谓我自然。
\zhushi{百姓不知君上之德淳厚,反以为己自当然也。}


\section{俗薄第十八}

大道废,有仁义。
\zhushi{大道之时,家有孝子,户有忠信,仁义不见也。大道废不用,恶逆生,乃有仁义可传道。}
智慧出,有大伪。
\zhushi{智慧之君贱德而贵言,贱质而贵文,下则应之以为大伪奸诈。}
六亲不和,有孝慈。
\zhushi{六纪绝,亲戚不合,乃有孝慈相牧养也。}
国家昏乱,有忠臣。
\zhushi{政令不明,上下相怨,邪僻争权,乃有忠臣匡正其君也。此言天下太平不知仁,人尽无欲不知廉,各自洁己不知贞。大道之世,仁义没,孝慈灭,犹日中盛明,众星失光。}


\section{还淳第十九}

绝圣
\zhushi{绝圣制作,反初守元。五帝垂象,仓颉作书,不如三皇结绳无文。}
弃智,
\zhushi{弃智慧,反无为。}
民利百倍。
\zhushi{农事修,公无私。}
绝仁弃义,
\zhushi{绝仁之见恩惠,弃义之尚华言。}
民复孝慈。
\zhushi{德化淳也。}
绝巧弃利,
\zhushi{绝巧者,诈伪乱真也。弃利者,塞贪路闭权门也。}
盗贼无有。
\zhushi{上化公正,下无邪私。}
此三者,
\zhushi{谓上三事所弃绝也。}
以为文不足,
\zhushi{以为文不足者,文不足以教民。}
故令有所属。
\zhushi{当如下句。}
见素抱朴,
\zhushi{见素者,当抱素守真,不尚文饰也。抱朴者,当抱其质朴,以示下,故可法则。}
少私寡欲。
\zhushi{少私者,正无私也。寡欲者,当知足也。}


\section{异俗第二十}

绝学
\zhushi{绝学不真,不合道文。}
无忧。
\zhushi{除浮华则无忧患也。}
唯之与阿,相去几何。
\zhushi{同为应对而相去几何。疾时贱质而贵文。}
善之与恶,相去若何。
\zhushi{善者称知其所穷极也。}
漂兮若无所止。
\zhushi{我独漂漂,若飞若扬,无所止也,志意在神域也。}
众人皆有以,
\zhushi{以,有为也。}
而我独顽
\zhushi{我独无为。}
似鄙。
\zhushi{鄙,似若不逮也。}
我独异于人
\zhushi{我独与人异也。}
而贵食母。
\zhushi{食,用也。母,道也。我独贵用道也。}


\section{虚心第二十一}

孔德之容,
\zhushi{孔,大也。有大德之人,无所不容,能受垢浊,处谦卑也。}
唯道是从。
\zhushi{唯,独也。大德之人,不随世俗所行,独从于道也。}
道之为物,唯恍唯忽。
\zhushi{道之于万物,独恍忽往来,于其无所定也。}
忽兮恍兮,其中有象;
\zhushi{道唯忽恍无形,之中独有万物法象。}
恍兮忽兮,其中有物。
\zhushi{道唯恍忽,其中有一,经营生化,因气立质。}
窈兮冥兮,其中有精,
\zhushi{道唯窈冥无形,其中有精实,神明相薄,阴阳交会也。}
其精甚真,
\zhushi{言存精气,其妙甚真,非有饰也。}
其中有信。
\zhushi{道匿功藏名,其信在中也。}
自古及今,其名不去,
\zhushi{自,从也。自古至今,道常在不去。}
以阅众甫,
\zhushi{阅,禀也。甫,始也。言道禀与,万物始生,从道受气。}
吾何以知众甫之然哉。
\zhushi{吾何以知万物从道受气。}
以此。
\zhushi{此,今也。以今万物皆得道精气而生,动作起居,非道不然。}


\section{益谦第二十二}

曲则全,
\zhushi{曲己从众,不自专,则全其身也。}
枉则直,
\zhushi{枉,屈己而伸人,久久自得直也。}
洼则盈,
\zhushi{地洼下,水流之;人谦下,德归之。}
弊则新,
\zhushi{自受弊薄,后己先人,天下敬之,久久自新也。}
少则得,
\zhushi{自受取少则得多也,天道佑谦,神明托虚。}
多则惑。
\zhushi{财多者,惑于所守,学多者,惑于所闻。}
是以圣人抱一为天下式。
\zhushi{抱,守也。式,法也。圣人守一,乃知万事,故能为天下法式也。}
不自见故明,
\zhushi{圣人不以其目视千里之外也,乃因天下之目以视,故能明达也。}
不自是故彰,
\zhushi{圣人不自以为是而非人,故能彰显于世。}
不自伐故有功,
\zhushi{伐,取也。圣人德化流行,不自取其美,故有功于天下。}
不自矜故长。
\zhushi{矜,大也。圣人不自贵大,故能久不危。}
夫唯不争,故天下莫能与之争。
\zhushi{此言天下贤与不肖,无能与不争者争也。}
古之所谓曲则全者,岂虚言哉。
\zhushi{传古言,曲从则全身,此言非虚妄也。}
诚全而归之。
\zhushi{诚,实也。能行曲从者,实其肌体,归之于父母,无有伤害也。}


\section{虚无第二十三}

希言自然。
\zhushi{希言者,谓爱言也。爱言者,自然之道。}
故飘风不终朝,骤雨不终日。
\zhushi{飘风,疾风也。骤雨,暴雨也。言疾不能长,暴不能久也。}
孰为此者?天地。
\zhushi{孰,谁也。谁为此飘风暴雨者乎?天地所为。}
天地尚不能久,
\zhushi{不能终于朝暮也。}
而况于人乎?
\zhushi{天地至神合为飘风暴雨,尚不能使终朝至暮,何况人欲为暴卒乎。}
故从事于道者,
\zhushi{从,为也。人为事当如道安静,不当如飘风骤雨也。}
道者同于道,
\zhushi{道者,谓好道人也。同于道者,所谓与道同也。}
德者同于德,
\zhushi{德者,谓好德之人也。同于德者,所谓与德同也。}
失者同于失。
\zhushi{失,谓任己而失人也。同于失者,所谓与失同也。}
同于道者,道亦乐得之。
\zhushi{与道同者,道亦乐得之也。}
同于德者,德亦乐得之,
\zhushi{与德同者,德亦乐得之也。}
同于失者,失亦乐失之。
\zhushi{与失同者,失亦乐失之也。}
信不足焉,
\zhushi{君信不足于下,下则应君以不信也。}
有不信焉。
\zhushi{此言物类相归,同声相应,同气相求。云从龙,风从虎,水流湿,火就燥,自然之类也。}


\section{苦恩第二十四}

跂者不立,
\zhushi{跂,进也。谓贪权慕名,进取功荣,则不可久立身行道也。}
跨者不行,
\zhushi{自以为贵而跨于人,众共蔽之,使不得行。}
自见者不明,
\zhushi{人自见其形容以为好,自见其所行以为应道,殊不知其形丑,操行之鄙。}
自是者不彰,
\zhushi{自以为是而非人,众共蔽之,使不得彰明。}
自伐者无功,
\zhushi{所谓辄自伐取其功美,即失有功于人也。}
自矜者不长。
\zhushi{好自矜大者,不可以长久。}
其其于道也,曰:馀食赘行。
\zhushi{赘,贪也。使此自矜伐之人,在治国之道,日赋敛馀禄食以为贪行。}
物或恶之。
\zhushi{此人在位,动欲伤害,故物无有不畏恶之者。}
故有道者不处也。
\zhushi{言有道之人不居其国也。}


\section{象元第二十五}

有物混成,先天地生。
\zhushi{谓道无形,混沌而成万物,乃在天地之前。}
寂兮寥兮,独立而不改,
\zhushi{寂者,无音声。寥者,空无形。独立者,无匹双。不改者,化有常。}
周行而不殆,
\zhushi{道通行天地,无所不入,在阳不焦,托荫不腐,无不贯穿,而不危怠也。}
可以为天下母。
\zhushi{道育养万物精气,如母之养子。}
吾不知其名,字之曰道,
\zhushi{我不见道之形容,不知当何以名之,见万物皆从道所生,故字之曰道。}
强为之名曰大。
\zhushi{不知其名,强曰大者,高而无上,罗而无外,无不包容,故曰大也。}
大曰逝,
\zhushi{其为大,非若天常在上,非若地常在下,乃复逝去,无常处所也。}
逝曰远,
\zhushi{言远者,穷乎无穷,布气天地,无所不通也。}
远曰反。
\zhushi{言其远不越绝,乃复反在人身也。}
故道大,天大,地大,王亦大。
\zhushi{道大者,包罗天地,无所不容也。天大者,无所不盖也。地大者,无所不载也。王大者,无所不制也。}
域中有四大,
\zhushi{四大,道、天、地、王也。凡有称有名,则非其极也。言道则有所由,有所由然后谓之为道,然则是道称中之大也,不若无称之大也,无称不可而得为名,曰域也。天地王皆在乎无称之内也,故曰域中有四大者也。}
而王居其一焉。
\zhushi{八极之内有四大,王居其一也。}
人法地,
\zhushi{人当法地安静和柔也,种之得五谷,掘之得甘泉,劳而不怨也,有功而不制也。}
地法天,
\zhushi{天湛泊不动,施而不求报,生长万物,无所收取。}
天法道,
\zhushi{道清静不言,阴行精气,万物自成也。}
道法自然。
\zhushi{道性自然,无所法也。}


\section{重德第二十六}

重为轻根,
\zhushi{人君不重则不尊,治身不重则失神,草木之花叶轻,故零落,根重故长存也。}
静为躁君。
\zhushi{人君不静则失威,治身不静则身危,龙静故能变化,虎躁故夭亏也。躁早报反}
是以圣人终日行,不离辎重。
\zhushi{辎,静也。圣人终日行道,不离其静与重也。离音利辎侧基反重直用反}
虽有荣观,燕处超然。
\zhushi{荣观,谓宫𨵗。燕处,后妃所居也。超然,逺避而不处也。观古乱反}
奈何万乘之主
\zhushi{奈何者,疾时主伤痛之辞。万乗之主谓,王乗绳证反}
而以身轻天下?
\zhushi{王者至尊,而以其身行轻躁乎。疾时王奢恣轻淫也。}
轻则失臣,
\zhushi{王者轻滔则失其臣,治身轻淫则失其精。}
躁则失君。
\zhushi{王者行躁疾则失其君位,治身躁疾则失其精神也。}


\section{巧用第二十七}

善行无辙迹,
\zhushi{善行道者求之于身,不下堂,不出门,故无辙迹。}
善言无瑕谪,
\zhushi{善言谓择言而出之,则无瑕疵谪过于天下。}
善计不用筹策,
\zhushi{善以道计事者,则守一不移,所计不多,则不用筹策而可知也。}
善闭无关楗而不可开
\zhushi{善以道闭情欲、守精神者,不如门户有关楗可得开。楗其偃反}
善结无绳约而不可解。
\zhushi{善以道结事者,乃可结其心,不如绳索可得解也。}
是以圣人常善救人,
\zhushi{圣人所以常教人忠孝者,欲以救人性命。}
故无弃人;
\zhushi{使贵贱各得其所也。}
常善救物,
\zhushi{圣人所以常教民顺四时者,欲以救万物之残伤。}
故无弃物。
\zhushi{圣人不贱名而贵玉视之如一。}
是谓袭明。
\zhushi{圣人善救人物,是谓袭明大道。}
故善人者,不善人之师;
\zhushi{人之行善者,圣人即以为人师。}
不善人者,善人之资。
\zhushi{资,用也。人行不善者,圣人犹教导使为善,得以给用也。}
不贵其师,
\zhushi{独无辅也。不爱其资无所使也。}
虽智大迷,
\zhushi{虽自以为智。言此人乃大迷惑。}
是谓要妙。
\zhushi{能通此意,是谓知微妙要道也。}


\section{反朴第二十八}

知其雄,守其雌,为天下溪。
\zhushi{雄以喻尊,神也。}


\section{无为第二十九}

将欲取天下
\zhushi{欲为天下主也。}
而为之,
\zhushi{欲以有为治民。}
吾见其不得已。
\zhushi{我见其不得天道人心已明矣,天道恶烦浊,人心恶多欲。}
天下神器,不可为也。
\zhushi{器,物也。人乃天下之神物也,神物好安静,不可以有为治。}
为者败之,
\zhushi{以有为治之,则败其质性。}
执者失之。
\zhushi{强执教之,则失其情实,生于诈伪也。}
故物或行或随,
\zhushi{上所行,下必随之也。}
或呴或吹,
\zhushi{歔,温也。吹,寒也。有所温必有所寒也。}
或强或羸,
\zhushi{有所强大,必有所羸弱也。}
或载或隳。
\zhushi{载,安也。隳,危也。有所安必有所危,明人君不可以有为治国与治身也。}
是以圣人去甚,去奢,去泰。
\zhushi{甚谓贪淫声色。奢谓服饰饮食。泰谓宫室台榭。去此三者,处中和,行无为,则天下自化。}


\section{俭武第三十}

以道佐人主者,
\zhushi{谓人主能以道自辅佐也。}
不以兵强天下。
\zhushi{以道自佐之主,不以兵革,顺天任德,敌人自服。}
其事好还。
\zhushi{其举事好还自责,不怨于人也。}
师之所处,荆棘生焉。
\zhushi{农事废,田不修。}
大军之后,必有凶年。
\zhushi{天应之以恶气,即害五谷,尽伤人也。}
善有果而已,
\zhushi{善用兵者,当果敢而已,不美之。}
不敢以取强。
\zhushi{不以果敢取强大之名也。}
果而勿矜
\zhushi{当果敢谦卑,勿自矜大也。}
果而勿伐,
\zhushi{当果敢推让,勿自伐取其美也。}
果而勿骄,
\zhushi{骄,欺也。果敢勿以骄欺人。}
果而不得已,
\zhushi{当过果敢至诚,不当逼迫不得已也。}
果而勿强
\zhushi{果敢勿以为强兵、坚甲以欺凌人也。}
物壮则老,
\zhushi{草木壮极则枯落,人壮极则衰老也。言强者不可以久。}
是谓不道。
\zhushi{枯老者,坐不行道也。}
不道早已。
\zhushi{不行道者早死。}


\section{偃武第三十一}

处之。
\zhushi{上将军居右,丧礼尚右,死人贵阴也。}
杀人之众,以哀悲泣之;
\zhushi{伤己德薄,不能以道化人,而害无辜之民。}
战胜,以丧礼处之。
\zhushi{古者战胜,将军居丧主礼之位,素服而哭之,明君子贵德而贱兵,不得以而诛不祥,心不乐之,比于丧也,知后世用兵不已故悲痛之。}


\section{圣德第三十二}

道常无名,
\zhushi{道能阴能阳,能弛能张,能存能亡,故无常名也。}
朴虽小,天下莫敢臣。
\zhushi{道朴虽小,微妙无形,天下不敢有臣使道者也。}
侯王若能守之,万物将自賔。
\zhushi{侯王若能守道无为,万物将自宾,服从于德也。}
天地相合,以降甘露,
\zhushi{侯王动作能与天相应和,天即降下甘露善瑞也。}
民莫之令而自均。
\zhushi{天降甘露善瑞,则万物莫有教令之者,皆自均调若一也。}
始制有名,
\zhushi{始,道也。有名,万物也。道无名能制于有名,无形,能制于有形也。}
名亦既有,
\zhushi{既,尽也。有名之物,尽有情欲,叛道离德,故身毁辱也。}
夫亦将知之。
\zhushi{人能法道行德,天亦将自知之。}
知之,可以不殆。
\zhushi{天知之,则神灵佑助,不复危怠。}
譬道之在天下,犹川谷之与江海。
\zhushi{譬言道之在天下,与人相应和,如川谷与江海相流通也。}


\section{辩德第三十三}

知人者智,
\zhushi{能知人好恶,是为智。}
自知者明。
\zhushi{人能自知贤与不肖,是为反听无声,内视无形,故为明也。}
胜人者有力,
\zhushi{能胜人者,不过以威力也。}
自胜者强。
\zhushi{人能自胜己情欲,则天下无有能与己争者,故为强也。}
知足者富,
\zhushi{人能知足,则长保福禄,故为富也。}
强行者有志,
\zhushi{人能强力行善,则为有意于道,道亦有意于人。}
不失其所者乆,
\zhushi{人能自节养,不失其所受天之精气,则可以长久。}
死而不亡者寿。
\zhushi{目不妄视,耳不妄听,口不妄言,则无怨恶于天下,故长寿。}


\section{任成第三十四}

大道泛兮,
\zhushi{言道泛泛,若浮若沉,若有若无,视之不见,说之难殊。泛音泛}
其可左右。
\zhushi{道可左右,无所不宜。}
万物恃之而生,
\zhushi{恃,待也。万物皆恃道而生。}
而不辞,
\zhushi{道不辞谢而逆止也。}
功成不名有,
\zhushi{有道不名其有功也。}
爱养万物而不为主。
\zhushi{道虽爱养万物,不如人主有所收取。}
常无欲,可名于小。
\zhushi{道匿德藏名,怕然无为,似若微小也。}
万物归焉而不为主,
\zhushi{万物皆归道受气,道非如人主有所禁止也。}
可名为大。
\zhushi{万物横来横去,使名自在,故不若于大也。}
是以圣人终不为大,
\zhushi{圣人法道匿德藏名,不为满大。}
故能成其大。
\zhushi{圣人以身师导,不言而化,万事修治,故能成其大。}


\section{仁德第三十五}

执大象,天下往。
\zhushi{执,守也。象,道也。圣人守大道,则天下万民移心归往之也。治身则天降神明,往来于己也。}
往而不害,安、平、太。
\zhushi{万物归往而不伤害,则国家安寕而致太平矣。治身不害神明,则身安而大寿也。}
乐与饵,过客止,
\zhushi{饵,美也。过客,一也。人能乐美于道,则一留止也。一者,去盈而处虚,忽忽如过客。}
道之出口,淡乎其无味,
\zhushi{道出入于口,淡淡非如五味有酸咸苦甘辛也。}
视之不足见,
\zhushi{足,德也。道无形,非若五色有青黄赤白黑可得见也。}
听之不足闻,
\zhushi{道非若五音有宫商角徵羽可得听闻也。}
用之不足既。
\zhushi{用道治国,则国安民昌。治身则寿命延长,无有既尽时也。}


\section{微明第三十六}

将欲歙之,必固张之。
\zhushi{先开张之者,欲极其奢淫。}
将欲弱之,必固强之。
\zhushi{先强大之者,欲使遇祸患。}
将欲废之,必固兴之。
\zhushi{先兴之者,欲使其骄危。}
将欲夺之,必固与之。
\zhushi{先与之者,欲极其贪心。}
是谓微明。
\zhushi{此四事,其道微,其效明也。}
柔弱胜刚强。
\zhushi{柔弱者久长,刚强者先亡也。}
鱼不可脱于渊,
\zhushi{鱼脱于渊,谓去刚得柔,不可复制焉。}
国之利器,不可以示人。
\zhushi{利器者,谓权道也。治国权者,不可以示执事之臣也。治身道者,不可以示非其人也。}


\section{为政第三十七}

道常无为而无不为。
\zhushi{道以无为为常也。}
侯王若能守之,万物将自化。
\zhushi{言侯王若能守道,万物将自化效于己也。}
化而欲作,吾将镇之以无名之朴。
\zhushi{吾,身也。无明之朴,道德也。万物已化效于己也。复欲作巧伪者,侯王当身镇抚以道德也。}
无名之朴,夫亦将无欲。不欲以静,
\zhushi{言侯王镇抚以道德,民亦将不欲,故当以清静导化之也。}
天下将自定。
\zhushi{能如是者,天下将自正定也。}


\section{论德第三十八}

上德不德,
\zhushi{上德,谓太古无名号之君,德大无上,故言上德也。不德者,言其不以德教民,因循自然,养人性命,其德不见,故言不德也。}
是以有德。
\zhushi{言其德合于天地,和气流行,民德以全也。}
下德不失德,
\zhushi{下德,谓号谥之君,德不及上德,故言下德也。不失德者,其德可见,其功可称也。}
是以无德。
\zhushi{以有名号及其身故。}
上德无为
\zhushi{谓法道安静,无所施为也。}
而无以为,
\zhushi{言无以名号为也。}
下德为之
\zhushi{言为教令,施政事也。}
而有以为。
\zhushi{言以为己取名号也}
上仁为之
\zhushi{上仁谓行仁之君,其仁无上,故言上仁。为之者,为人恩也。}
而无以为,
\zhushi{功成事立,无以执为。}
上义为之
\zhushi{为义以断割也。}
而有以为。
\zhushi{动作以为己,杀人以成威,贼下以自奉也。}
上礼为之
\zhushi{谓上礼之君,其礼无上,故言上礼。为之者,言为礼制度,序威仪也。}
而莫之应,
\zhushi{言礼华盛实衰,饰伪烦多,动则离道,不可应也。}
则攘臂而扔之。
\zhushi{言礼烦多不可应,上下忿争,故攘臂相仍引。}
故失道而后德,
\zhushi{言道衰而德化生也。}
失德而后仁,
\zhushi{言德衰而仁爱见也。}
失仁而后义,
\zhushi{言仁衰而分义明也。}
失义而后礼。
\zhushi{言义衰则失礼聘,行玉帛也。}
夫礼者,忠信之薄
\zhushi{言礼废本治末,忠信日以衰薄。}
而乱之首。
\zhushi{礼者贱质而贵文,故正直日以少,邪乱日以生。}
前识者,道之华
\zhushi{不知而言知为前识,此人失道之实,得道之华。}
而愚之始。
\zhushi{言前识之人,愚暗之倡始也。}
是以大丈夫处其厚,
\zhushi{大丈夫谓得道之君也。处其厚者,谓处身于敦朴。}
不居其薄,
\zhushi{不处身违道,为世烦乱也。}
处其实,
\zhushi{处忠信也。}
不居其华。
\zhushi{不尚华言也。}
故去彼取此。
\zhushi{去彼华薄,取此厚实。}


\section{法本第三十九}


昔之得一者:
\zhushi{昔,往也。一,无为,道之子也。}
天得一以清,
\zhushi{言天得一故能垂象清明。}
地得一以宁,
\zhushi{言地得一故能安静不动摇。}
神得一以灵,
\zhushi{言神得一故能变化无形。}
谷得一以盈,
\zhushi{言谷得一故能盈满而不绝也}
万物得一以生,
\zhushi{言万物皆须道以生成也。}
侯王得一以为天下贞。
\zhushi{言侯王得一故能为天下平正}
其致之。
\zhushi{致,诫也。谓下六事也。}
天无以清将恐裂,
\zhushi{言天当有阴阳弛张,昼夜更用,不可但欲清明无已时,将恐分裂不为天。}
地无以宁将恐发,
\zhushi{言地当有高下刚柔,节气五行,不可但欲安静无已时,将恐发泄不为地。}
神无以灵将恐歇,
\zhushi{言神当有王相囚死休废,不可但欲灵变无已时,将恐虚歇不为神。}
谷无以盈将恐竭,
\zhushi{言谷当有盈缩虚实,不可但欲盈满无已时,将恐枯竭不为谷。}
万物无以生将恐灭,
\zhushi{言万物当随时生死,不可但欲长生无已时,将恐灭亡不为物。}
侯王无以贵高将恐蹶。
\zhushi{言侯王当屈己以下人,汲汲求贤,不可但欲贵高于人无已时,将恐颠蹶失其位。}
故贵以贱为本,
\zhushi{言必欲尊贵,当以薄贱为本,若禹稷躬稼,舜陶河滨,周公下白屋也。}
高以下为基
\zhushi{言必欲尊贵,当以下为本基,犹筑墙造功,因卑成高,下不坚固,后必倾危。}
是以侯王自谓孤、寡、不毂。
\zhushi{孤寡喻孤独,不毂喻不能如车毂为众辐所凑。}
此非以贱为本邪?
\zhushi{言侯王至尊贵,能以孤寡自称,此非以贱为本乎,以晓人?}
非乎!
\zhushi{嗟叹之辞。}
故致数舆无舆,
\zhushi{致,就也。言人就车数之为辐、为轮、为毂、为衡、为舆,无有名为车者,故成为车,以喻侯王不以尊号自名,故能成其贵。}
不欲琭琭如玉,珞珞如石。
\zhushi{琭琭喻少,落落喻多,玉少故见贵,石多故见贱。言不欲如玉为人所贵,如石为人所贱,当处其中也。}

\section{去用第四十}

反者道之动,
\zhushi{反,本也。本者,道之所以动,动生万物,背之则亡也。}
弱者道之用。
\zhushi{柔弱者,道之所常用,故能常久。}
天下万物生于有,
\zhushi{天下万物皆从天地生,天地有形位,故言生于有也。}
有生于无。
\zhushi{天地神明,蜎飞蠕动,皆从道生。道无形,故言生于无也。此言本胜于华,弱胜于强,谦虚胜盈满也。}


\section{同异第四十一}

章上士闻道,勤而行之。
\zhushi{上士闻道,自勤苦竭力而行之。}
中士闻道,若存若亡。
\zhushi{中士闻道,治身以长存,治国以太平,欣然而存之,退见财色荣誉,惑于情欲,而复亡之也。}
下士闻道,大笑之。
\zhushi{下士贪狠多欲,见道柔弱,谓之恐惧,见道质朴,谓之鄙陋,故大笑之。}
不笑不足以为道。
\zhushi{不为下士所笑,不足以名为道。}
故建言有之:
\zhushi{建,设也。设言以有道,当如下句。}
明道若昧,
\zhushi{明道之人,若暗昧无所见。}
进道若退,
\zhushi{进取道者,若退不及。}
夷道若纇。
\zhushi{夷,平也。大道之人不自别殊,若多比类也。}
上德若谷,
\zhushi{上德之人若深谷,不耻垢浊也。}
大白若辱,
\zhushi{大洁白之人若污辱,不自彰显。}
广德若不足,
\zhushi{德行广大之人,若愚顽不足也。}
建德若偷,
\zhushi{建设道德之人,若可偷引使空虚也。}
质真若渝,
\zhushi{质朴之人,若五色有渝浅不明也。}
大方无隅,
\zhushi{大方正之人,无委屈廉隅。}
大器晚成,
\zhushi{大器之人,若九鼎瑚琏,不可卒成也。}
大音希声,
\zhushi{大音犹雷霆待时而动,喻当爱气希言也。}
大象无形,
\zhushi{大法象之人,质朴无形容。}
道隐无名。
\zhushi{道潜隐,使人无能指名也。}
夫惟道,善贷且成。
\zhushi{成,就也。言道善禀贷人精气,且成就之也。}


\section{道化第四十二}

道生一,
\zhushi{道使所生者一也。}
一生二,
\zhushi{一生阴与阳也。}
二生三,
\zhushi{阴阳生和、清、浊三气,分为天地人也。}
三生万物。
\zhushi{天地人共生万物也,天施地化,人长养之也。}
万物负阴而抱阳,
\zhushi{万物无不负阴而向阳,回心而就日。}
冲气以为和。
\zhushi{万物中皆有元气,得以和柔,若胸中有藏,骨中有髓,草木中有空虚与气通,故得久生也。}
人之所恶,惟孤、寡、不谷,而王公以为称。
\zhushi{孤寡不毂者,不祥之名,而王公以为称者,处谦卑,法空虚和柔。}
故物或损之而益,
\zhushi{引之不得,推之必还。}
或益之而损。
\zhushi{夫增高者志崩,贪富者致患。}
人之所教,
\zhushi{谓众人所教,去弱为强,去柔为刚。}
我亦教之。
\zhushi{言我教众人,使去强为弱,去刚为柔。}
强梁者不得其死,
\zhushi{强粱者,谓不信玄妙,背叛道德,不从经教,尚势任力也。不得其死者,为天命所绝,兵刃所伐,王法所杀,不得以寿命死。}
吾将以为教父。
\zhushi{父,始也。老子以强梁之人为教,诫之始也。}


\section{偏用第四十三}

天下之至柔,驰骋天下之至坚。
\zhushi{至柔者,水也。至坚者,金石也。水能贯坚入刚,无所不通。}
无有入无间。
\zhushi{无有谓道也。道无形质,故能出入无间,通神明济群生也。}
吾是以知无为之有益。
\zhushi{吾见道无为而万物自化成,是以知无为之有益于人也。}
不言之教,
\zhushi{法道不言,师之以身。}
无为之益,
\zhushi{法道无为,治身则有益于精神,治国则有益于万民,不劳烦也。}
天下希及之。
\zhushi{天下,人主也。希能有及道无为之治身治国也。}


\section{立戒第四十四}

名与身孰亲。
\zhushi{名遂则身退也。}
身与货孰多。
\zhushi{财多则害身也。}
得与亡孰病。
\zhushi{好得利则病于行也。}
甚爱必大费,
\zhushi{甚爱色,费精神。甚爱财,遇祸患。所爱者少,所亡者多,故言大费。}
多藏必厚亡。
\zhushi{生多藏于府库,死多藏于丘墓。生有攻劫之忧,死有掘冢探柩之患。}
知足不辱,
\zhushi{知足之人绝利去欲,不辱于身。}
知止不殆,
\zhushi{知可止,则财利不累于身,声色不乱于耳目,则身不危殆也。}
可以长久。
\zhushi{人能知止足则福禄在己,治身者,神不劳;治国者,民不扰,故可长久。}


\section{洪德第四十五}

大成若缺,
\zhushi{谓道德大成之君也。若缺者,灭名藏誉,如毁缺不备也。}
其用不弊,
\zhushi{其用心如是,则无敝尽时也。}
大盈若冲,
\zhushi{谓道德大盈满之君也。若冲者,贵不敢骄也,富不敢奢也。}
其用不穷。
\zhushi{其用心如是,则无穷尽时也。}
大直若屈,
\zhushi{大直,谓修道法度正直如一也。若屈者,不与俗人争,若可屈折。}
大巧若拙,
\zhushi{大巧谓多才术也。若拙者,亦不敢见其能。}
大辩若讷。
\zhushi{大辩者,智无疑。若讷者,口无辞。}
躁胜寒,
\zhushi{胜,极也。春夏阳气躁疾于上,万物盛大,极则寒,寒则零落死亡也。言人不当刚躁也。}
静胜热,
\zhushi{秋冬万物静于黄泉之下,极则热,热者生之源。}
清静能为天下正。
\zhushi{能清静则为天下之长,持身正则无终已时也。}


\section{俭欲第四十六}

天下有道,
\zhushi{谓人主有道也。}
却走马以粪,
\zhushi{粪者,粪田也。兵甲不用,却走马治农田,治身者却阳精以粪其身。}
天下无道,
\zhushi{谓人主无道也。}
戎马生于郊。
\zhushi{战伐不止,戎马生于郊境之上,久不还也。}
罪莫大于可欲。
\zhushi{好淫色也。}
祸莫大于不知足,
\zhushi{富贵不能自禁止也。}
咎莫大于欲得。
\zhushi{欲得人物,利且贪也。}
故知足之足,
\zhushi{守真根也。}
常足。
\zhushi{无欲心也。}


\section{鉴远第四十七}

不出户知天下,
\zhushi{圣人不出户以知天下者,以己身知人身,以己家知人家,所以见天下也。}
不窥牖见天道,
\zhushi{天道与人道同,天人相通,精气相贯。人君清净,天气自正,人君多欲,天气烦浊。吉凶利害,皆由于己。}
其出弥远,其知弥少。
\zhushi{谓去其家观人家,去其身观人身,所观益远,所见益少也。}
是以圣人不行而知,
\zhushi{圣人不上天,不入渊,能知天下者,以心知之也。}
不见而名,
\zhushi{上好道,下好德;上好武,下好力。圣人原小知大,察内知外。}
不为而成。
\zhushi{上无所为,则下无事,家给人足,万物自化就也。}


\section{忘知第四十八}

为学日益,
\zhushi{学谓政教礼乐之学也。日益者,情欲文饰日以益多。}
为道日损。
\zhushi{道谓之自然之道也。日损者,情欲文饰日以消损。}
损之又损,
\zhushi{损情欲也。又损之,所以渐去。}
以至于无为,
\zhushi{当恬淡如婴儿,无所造为也。}
无为而无不为。
\zhushi{情欲断绝,德于道合,则无所不施,无所不为也。}
取天下常以无事,
\zhushi{取,治也。治天下当以无事,不当以劳烦也。}
及其有事,不足以取天下。
\zhushi{及其好有事,则政教烦,民不安,故不足以治天下也。}


\section{任德四十九}

圣人无常心,
\zhushi{圣人重改更,贵因循,若自无心。}
以百姓心为心。
\zhushi{百姓心之所便,圣人因而从之。}
善者吾善之,
\zhushi{百姓为善,圣人因而善之。}
不善者吾亦善之,
\zhushi{百姓虽有不善者,圣人化之使善也。}
德善。
\zhushi{百姓德化,圣人为善}
信者吾信之,
\zhushi{百姓为信,圣人因而信之。}
不信者吾亦信之,
\zhushi{百姓为不信,圣人化之为信者也。}
德信。
\zhushi{百姓德化,圣人以为信。}
圣人在天下怵怵,
\zhushi{圣人在天下怵怵常恐怖,富贵不敢骄奢。}
为天下浑其心。
\zhushi{言圣人为天下百姓混浊其心,若愚暗不通也。}
百姓皆注其耳目,
\zhushi{注,用也。百姓皆用其耳目为圣人视听也。}
圣人皆孩之。
\zhushi{圣人爱念百姓如婴孩赤子,长养之而不责望其报。}


\section{贵生第五十}

出生入死。
\zhushi{出生,谓情欲出五内,魂静魄定,故生。入死,谓情欲入于胸臆,精劳神惑,故死。}
生之徒十有三,死之徒死十有三,
\zhushi{言生死之类各有十三,谓九窍四关也。其生也目不妄视,耳不妄听,鼻不妄嗅,口不妄言,味,手不妄持,足不妄行,精神不妄施。其死也反是也。}
人之生,动之死地十有三。
\zhushi{人知求生,动作反之十三死也。}
夫何故,
\zhushi{问何故动之死地也。}
以其求生之厚。
\zhushi{所以动之死地者,以其求生活之事太厚,违道忤天,妄行失纪。}
盖以闻善摄生者,
\zhushi{摄,养也。}
路行不遇兕虎,
\zhushi{自然远离,害不干也。}
入军不披甲兵,
\zhushi{不好战以杀人。}
兕无投其角,虎无所措爪,兵无所容其刃。
\zhushi{养生之人,兕虎无由伤,兵刃无从加之也。}
夫何故,
\zhushi{问兕虎兵甲何故不加害之。}
以其无死地。
\zhushi{以其不犯十三之死地也。言神明营护之,此物不敢害。}


\section{养德第五十一}

道生之,
\zhushi{道生万物。}
德畜之,
\zhushi{德,一也。一主布气而蓄养}
物形之,
\zhushi{一为万物设形像也。}
势成之。
\zhushi{一为万物作寒暑之势以成之。}
是以万物莫不尊道而贵德。
\zhushi{道德所为,无不尽惊动,而尊敬之。}
道之尊,德之贵,夫莫之命而常自然。
\zhushi{道一不命召万物,而常自然应之如影响。}
故道生之,德畜之,长之育之,成之孰之,养之覆之。
\zhushi{道之于万物,非但生而已,乃复长养、成孰、覆育,全其性命。人君治国治身,亦当如是也。}
生而不有,
\zhushi{道生万物,不有所取以为利也。}
为而不恃,
\zhushi{道所施为,不恃望其报也。}
长而不宰,
\zhushi{道长养万物,不宰割以为利也。}
是谓玄德。
\zhushi{道之所行恩德,玄暗不可得见。}


\section{归元第五十二}

天下有始,
\zhushi{始有道也。}
以为天下母。
\zhushi{道为天下万物之母}
既知其母,复知其子,
\zhushi{子,一也。既知道己,当复知一也。}
既知其子,复守其母,
\zhushi{己知一,当复守道反无为也。}
没身不殆。
\zhushi{不危殆也。}
塞其兑,
\zhushi{兑,目也。目不妄视也。}
闭其门,
\zhushi{门,口也。使口不妄言}
终身不勤。
\zhushi{人当塞目不妄视,闭口不妄言,则终生不勤苦。}
开其兑,
\zhushi{开目视情欲也。}
济其事,
\zhushi{济,益也。益情欲之事。}
终身不救。
\zhushi{祸乱成也。}
见小曰明,
\zhushi{萌芽未动,祸乱未见为小,昭然独见为明。}
守柔日强。
\zhushi{守柔弱,日以强大也。}
用其光,
\zhushi{用其目光于外,视时世之利害。}
复归其明。
\zhushi{复当返其光明于内,无使精神泄也。}
无遗身殃,
\zhushi{内视存神,不为漏失。}
是谓习常。
\zhushi{人能行此,是谓修习常道。}


\section{益证第五十三}

使我介然有知,行于大道。
\zhushi{介,大也。老子疾时王不行大道,故设此言。使我介然有知于政事,我则行于大道,躬行无为之化。}
唯施是畏。
\zhushi{唯,独也。独畏有所施为,恐失道意。欲赏善,恐伪善生;欲信忠恐诈忠起。}
大道甚夷,而民好径。
\zhushi{夷,平易也。径,邪、不平正也。大道甚平易,而民好从邪径也。}
朝甚除,
\zhushi{高台榭,宫室修。}
田甚芜,
\zhushi{农事废,不耕治。}
仓甚虚,
\zhushi{五谷伤害,国无储也。}
服文彩,
\zhushi{好饰伪,贵外华。}
带利剑,
\zhushi{尚刚强,武且奢。}
厌饮食,财货有馀,
\zhushi{多嗜欲,无足时。}
是谓盗夸。
\zhushi{百姓而君有馀者,是由劫盗以为服饰,持行夸人,不知身死家破,亲戚并随也。}
非道哉。
\zhushi{人君所行如是,此非道也。复言也哉者,痛伤之辞。}


\section{修观第五十四}

天下。
\zhushi{以修道之主,观不修道之主也。}
吾何以知天下之然哉,以此。
\zhushi{老子言,吾何知天下修道者昌,背道者亡。以此五事观而知之也。}


\section{玄符第五十五}

含德之厚,
\zhushi{谓含怀道德之厚也。}
比于赤子。
\zhushi{神明保佑含德之人,若父母之于赤子也。}
毒虫不螫,
\zhushi{蜂蠇蛇虺不螫。}
猛兽不据,玃鸟不搏。
\zhushi{赤子不害于物,物亦不害之。故太平之世,人无贵贱,仁心,有刺之物,还返其本,有毒之虫,不伤于人。}
骨弱筋柔而握固。
\zhushi{赤子筋骨柔弱而持物坚固,以其意心不移也。}
未知牝牡之合而峻作精之至也。
\zhushi{赤子未知男女会合而阴阳作怒者,由精气多之所致也。}
终日号而不哑,和之至也。
\zhushi{赤子从朝至暮啼号声不变易者,和气多之所至也。}
知和日常,
\zhushi{人能和气柔弱有益于人者,则为知道之常也。}
知常日明,
\zhushi{人能知道之常行,则日以明达于玄妙也。}
益生日祥,
\zhushi{祥,长也。言益生欲自生,日以长大。}
心使气日强。
\zhushi{心当专一和柔而神气实内,故形柔。而反使妄有所为,和气去于中,故形体日以刚强也。}
物壮则老,
\zhushi{万物壮极则枯老也。}
谓之不道,
\zhushi{枯老则不得道矣。}
不道早已。
\zhushi{不得道者早死。}


\section{玄德第五十六}

知者不言,
\zhushi{知者贵行不贵言也。}
言者不知。
\zhushi{驷不及舌,多言多患。}
塞其兑,闭其门,
\zhushi{塞闭之者,欲绝其源。}
挫其锐,
\zhushi{情欲有所锐为,当念道无为以挫止之。}
解其纷,
\zhushi{纷,结恨不休也。当念道恬怕以解释之。}
和其光,
\zhushi{虽有独见之明,当和之使暗昧,不使曜乱。}
同其尘,
\zhushi{不当自别殊也。}
是谓玄同。
\zhushi{玄,天也。人能行此上事,是谓与天同道也。}
故不可得而亲,
\zhushi{不以荣誉为乐,独立为哀。}
亦不可得而踈;
\zhushi{志静无欲,故与人无怨。}
不可得而利,
\zhushi{身不欲富贵,口不欲五味。}
亦不可得而害,
\zhushi{不与贪争利,不与勇争气。}
不可得而贵,
\zhushi{不为乱世主,不处暗君位。}
亦不可得而贱,
\zhushi{不以乘权故骄,不以失志故屈。}
故为天下贵。
\zhushi{其德如此,天子不得臣,诸侯不得屈,与世沉浮容身避害,故天下贵也。}


\section{淳风第五十七}

以正治国,
\zhushi{以,至也。天使正身之人,使有国也。}
以奇用兵,
\zhushi{奇,诈也。天使诈伪之人,使用兵也。}
以无事取天下。
\zhushi{以无事无为之人,使取天下为之主。}
吾何以知其然哉,以此。
\zhushi{此,今也。老子言,我何以知天意然哉,以今日所见知。}
天下多忌讳而民弥贫。
\zhushi{天下谓人主也。忌讳者防禁也。今烦则奸生,禁多则下诈,相殆故贫。}
民多利器,国家滋昏。
\zhushi{利器者,权也。民多权则视者眩于目,听者惑于耳,上下不亲,故国家昏乱。}
人多伎巧,奇物滋起。
\zhushi{人谓人君、百里诸侯也。多技巧,谓刻画宫观,雕琢章服,奇物滋起,下则化上,饰金镂玉,文绣彩色日以滋甚。}
法物滋彰,盗贼多有。
\zhushi{法物,好物也。珍好之物滋生彰著,则农事废,饥寒并至,而盗贼多有也。}
故圣人云:
\zhushi{谓下事也。}
我无为而民自化,
\zhushi{圣人言:我修道承天,无所改作,而民自化成也。}
我好静而民自正,
\zhushi{圣人言:我好静,不言不教,而民自忠正也。}
我无事而民自富,
\zhushi{我无徭役徵召之事,民安其业故皆自富也。}
我无欲而民自朴。
\zhushi{我常无欲,去华文,微服饰,民则随我为质朴也。圣人言:我修道守真,绝去六情,民自随我而清也。}


\section{顺化第五十八}

其政闷闷,
\zhushi{其政教宽大,闷闷昧昧,似若不明也。}
其民醇醇,
\zhushi{政教宽大,故民醇醇富厚,相亲睦也。}
其政察察,
\zhushi{其政教急疾,言决于口,听决于耳也。}
其民缺缺。
\zhushi{政教急疾。民不聊生。故缺缺日以踈薄。}
祸兮福所倚,
\zhushi{倚,因也。夫福因祸而生,人遭祸而能悔过责己,修道行善,则祸去福来。}
福兮祸所伏。
\zhushi{祸伏匿于福中,人得福而为骄恣,则福去祸来。}
孰知其极,
\zhushi{祸福更相生,谁能知其穷极时。}
其无正,
\zhushi{无,不也。谓人君不正其身,其无国也。}
正复为奇,
\zhushi{奇,诈也。人君不正,下虽正,复化上为诈也。}
善复为訞。
\zhushi{善人皆复化上为訞祥也。}
人之迷,其日固久。
\zhushi{言人君迷惑失正以来,其日已固久。}
是以圣人方而不割,
\zhushi{圣人行方正者,欲以率下,不以割截人也。}
廉而不害,
\zhushi{圣人廉清,欲以化民,不以伤害人也。今则不然,正己以害人也。}
直而不肆,
\zhushi{肆,申也。圣人虽直,曲己从人,不自申也。}
光而不曜。
\zhushi{圣人虽有独见之明,当如暗昧,不以曜乱人也。}


\section{守道第五十九}

治人,
\zhushi{谓人君治理人民。}
事天,
\zhushi{事,用也。当用天道,顺四时。}
莫若啬。
\zhushi{啬,爱惜也。治国者当爱民财,不为奢泰。治身者当爱精气,不为放逸。}
夫为啬,是谓早服。
\zhushi{早,先也。服,得也。夫独爱民财,爱精气,则能先得天道也。}
早服谓之重积德。
\zhushi{先得天道,是谓重积得于己也。}
重积德则无不克,
\zhushi{克,胜也。重积德于己,则无不胜。}
无不克则莫知其极,
\zhushi{无不克胜,则莫知有知己德之穷极也。}
莫知其极可以有国。
\zhushi{莫知己德者有极,则可以有社稷,为民致福。}
有国之母,可以长久。
\zhushi{国身同也。母,道也。人能保身中之道,使精气不劳,五神不苦,则可以长久。}
是谓深根固蒂,
\zhushi{人能以气为根,以精为蒂,如树根不深则拔,蒂不坚则落。言当深藏其气,固守其精,使无漏泄。}
长生久视之道。
\zhushi{深根固蒂者,乃长生久视之道。}


\section{居位第六十}

治大国者若烹小鲜。
\zhushi{鲜,鱼。烹小鱼不去肠、不去鳞、不敢挠,恐其糜也。治国烦则下乱,治身烦则精散。}
以道莅天下,其鬼不神。
\zhushi{以道德居位治天下,则鬼不敢以其精神犯人也。}
非其鬼不神,其神不伤人。
\zhushi{其鬼非无精神也,非不入正,不能伤自然之人。}
非其神不伤人,圣人亦不伤。
\zhushi{非鬼神不能伤害人。以圣人在位不伤害人,故鬼不敢干之也。}
夫两不相伤,
\zhushi{鬼与圣人俱两不相伤也。}
故德交归焉。
\zhushi{夫两不相伤,则人得治于阳,鬼神得治于阴,人得保全其性命,鬼得保其精神,故德交归焉。}


\section{谦德第六十一}

大国者下流,
\zhushi{治大国,当如居下流,不逆细微。}
天下之交,
\zhushi{大国,天下士民之所交会。}
天下之牝。
\zhushi{牝者,阴类也。柔谦和而不昌也。}
牝常以静胜牡,
\zhushi{女所以能屈男,阴胜阳,以,安静不先求之也。}
以静为下。
\zhushi{阴道以安静为谦下。}
故大国以下小国,则取小国,
\zhushi{能谦下之,则常有之。}
小国以下大国,则取大国。
\zhushi{此言国无大小,能持谦畜人,则无过失也。}
故或下以取,或下而取。
\zhushi{下者谓大国以下小国,小国以下大国,更以义相取。}
大国不过欲兼畜人,
\zhushi{大国不失下,则兼并小国而牧畜之。}
小国不过欲入事人。
\zhushi{使为臣仆。}
夫两者各得其所欲,大者宜为下。
\zhushi{大国小国各欲得其所,大国又宜为谦下}


\section{为道第六十二}

道者万物之奥,
\zhushi{奥,藏也。道为万物之藏,无所不容也。}
善人之宝,
\zhushi{善人以道为身宝,不敢违也。}
不善人之所保。
\zhushi{道者,不善人之保倚也。遭患逢急,犹知自悔卑下。}
美言可以市,
\zhushi{美言者独可于市耳。夫市交易而退,不相宜善言美语,求者欲疾得,卖者欲疾售也。}
尊行可以加入。
\zhushi{加,别也。人有尊贵之行,可以别异于凡人,未足以尊道。}
人之不善,何弃之有。
\zhushi{人虽不善,当以道化之。盖三皇之前,无有弃民,德化淳也。}
故立天子,置三公,
\zhushi{欲使教化不善之人。}
虽有拱璧以先驷马,不如坐进此道。
\zhushi{虽有美璧先驷马而至,故不如坐进此道。}
古之所以贵此道者,何不日以求得?
\zhushi{古之所以贵此道者,不日日远行求索,近得之于身。}
有罪以免耶,
\zhushi{有罪谓遭乱世,暗君妄行形诛,修道则可以解死,免于众也。}
故为天下贵。
\zhushi{道德洞远,无不覆济,全身治国,恬然无为,故可为天下贵也。}


\section{恩始第六十三}

为无为,
\zhushi{因成循故,无所造作。}
事无事,
\zhushi{预有备,除烦省事也。}
味无味。
\zhushi{深思远虑,味道意也。}
大小多少,
\zhushi{陈其戒令也。欲大反小,欲多反少,自然之道也。}
报怨以德。
\zhushi{修道行善,绝祸于未生也。}
图难于其易,
\zhushi{欲图难事,当于易时,未及成也。}
为大于其细。
\zhushi{欲为大事,必作于小,祸乱从小来也。}
天下难事必作于易,天下大事必作于细。
\zhushi{从易生难,从细生著。}
是以圣人终不为大,故能成其大。
\zhushi{处谦虚,天下共归之也。}
夫轻诺必寡信,
\zhushi{不重言也。}
多易必多难。
\zhushi{不慎患也。}
是以圣人犹难之,
\zhushi{圣人动作举事,犹进退,重难之,欲塞其源也。}
故终无难矣。
\zhushi{圣人终生无患难之事,犹避害深也}


\section{守微第六十四}

其安易持,
\zhushi{治身治国安静者,易守持也。}
其未兆易谋,
\zhushi{情欲祸患未有形兆时,易谋止也。}
其脆易破,
\zhushi{祸乱未动于朝,情欲未见于色,如脆弱易破除。}
其微易散。
\zhushi{其未彰著,微小易散去也。}
为之于未有,
\zhushi{欲有所为,当于未有萌芽之时塞其端也。}
治之于未乱。
\zhushi{治身治国于未乱之时,当豫闭其门也。}
合抱之木生于毫末;
\zhushi{从小成大。}
九层之台起于累土;
\zhushi{从卑立高。}
千里之行始于足下。
\zhushi{从近至远。}
为者败之,
\zhushi{有为于事,废于自然;有为于义,废于仁;有为于色,废于精神也。}
执者失之。
\zhushi{执利遇患,执道全身,坚持不得,推让反还。}
是以圣人无为故无败,
\zhushi{圣人不为华文,不为色利,不为残贼,故无败坏。}
无执故无失。
\zhushi{圣人有德以教愚,有财以与贫,无所执藏,故无所失于人也。}
民之从事,常于几成而败之。
\zhushi{从,为也。民之为事,常于功德几成,而贪位好名,奢泰盈满而自败之也。}
慎终如始,则无败事。
\zhushi{终当如始,不当懈怠。}
是以圣人欲不欲,
\zhushi{圣人欲人所不欲。人欲彰显,圣人欲伏光;人欲文饰,圣人欲质朴;人欲色,圣人欲于德。}
不贵难得之货;
\zhushi{圣人不眩为服,不贱石而贵玉。}
学不学,
\zhushi{圣人学人所不能学。人学智诈,圣人学自然;人学治世,圣人学治身;守道真也。}
复众人之所过;
\zhushi{众人学问反,过本为末,过实为华。复之者,使反本也。}
以辅万物之自然。
\zhushi{教人反本实者,欲以辅助万物自然之性也。}
而不敢为。
\zhushi{圣人动作因循,不敢有所造为,恐远本也。}


\section{淳德第六十五}

古之善为道者,非以明民,将以愚之。
\zhushi{说古之善以道治身及治国者,不以道教民明智巧诈也,将以道德教民,使质朴不诈伪。}
民之难治,以其智多。
\zhushi{民之所以难治者,以其智多而为巧伪。}
故以智治国,国之贼;
\zhushi{使智慧之人治国之政事,必远道德,妄作威福,为国之贼也。}
不以智治国,国之福。
\zhushi{不使智慧之人治国之政事,则民守正直,不为邪饰,上下相亲,君臣同力,故为国之福也。}
知此两者亦稽式。
\zhushi{两者谓智与不智也。常能智者为贼,不智者为福,是治身治国之法式也。}
常知稽式,是谓玄德。
\zhushi{玄,天也。能知治身及治国之法式,是谓与天同德也。}
玄德深矣,远矣,
\zhushi{玄德之人深不可测,远不可及也。}
与物反矣!
\zhushi{玄德之人与万物反异,万物欲益己,玄德施与人也。}
然后乃至于大顺。
\zhushi{玄德与万物反异,故能至大顺。顺天理也。}


\section{后己第六十六}

江海所以能为百谷王者,以其善下之,故能为百谷王。
\zhushi{江海以卑,故众流归之,若民归就王。以卑下,故能为百谷王也。}
是以欲上民,
\zhushi{欲在民之上也。}
必以言下之;
\zhushi{法江海处谦虚。}
欲先民,
\zhushi{欲在民之前也。}
必以身后之。
\zhushi{先人而后己也。}
是以圣人处上而民不重,
\zhushi{圣人在民上为主,不以尊贵虐下,故民戴而不为重。}
处前而民不害。
\zhushi{圣人在民前,不以光明蔽后,民亲之若父母,无有欲害之心也。}
是以天下乐推而不厌。
\zhushi{圣人恩深爱厚,视民如赤子,故天下乐推进以为主,无有厌也。}
以其不争,
\zhushi{天下无厌圣人时,是由圣人不与人争先后也。}
故天下莫能与之争。
\zhushi{言人皆有为,无有与吾争无为。}


\section{三宝第六十七}

天下皆谓我大,似不肖。
\zhushi{老子言:天下谓我德大,我则佯愚似不肖。}
夫唯大,故似不肖,
\zhushi{唯独名德大者为身害,故佯愚似若不肖。无所分别,无所割截,不贱人而自责。}
若肖久矣。
\zhushi{肖,善也。谓辨惠也。若大辨惠之人,身高自贵行察察之政所从来久矣。}
其细也夫。
\zhushi{言辨惠者唯如小人,非长者。}
我有三宝,持而保之。
\zhushi{老子言:我有三宝,抱持而保倚。}
一曰慈,
\zhushi{爱百姓若赤子。}
二曰俭,
\zhushi{赋敛若取之于己也。}
三曰不敢为天下先。
\zhushi{执谦退,不为倡始也。}
慈故能勇,
\zhushi{以慈仁,故能勇于忠孝也。}
俭故能广,
\zhushi{天子身能节俭,故民日用广矣。}
不敢为天下先,
\zhushi{不为天下首先。}
故能成器长。
\zhushi{成器长,谓得道人也。我能为得道人之长也。}
今舍慈且勇,
\zhushi{今世人舍慈仁,但为勇武。}
舍俭且广,
\zhushi{舍其俭约,但为奢泰。}
舍后且先,
\zhushi{舍其后己,但为人先。}
死矣!
\zhushi{所行如此,动入死地。}
夫慈以战则胜,以守则固。
\zhushi{夫慈仁者,百姓亲附,并心一意,故以战则胜敌,以守卫则坚固。}
天将救之,以慈卫之。
\zhushi{天将救助善人,必与慈仁之性,使能自营助也。}


\section{配天第六十八}

善为士者不武,
\zhushi{言贵道德,不好武力也。}
善战者不怒,
\zhushi{善以道战者,禁邪于胸心,绝祸于未萌,无所诛怒也。}
善胜敌者不与,
\zhushi{善以道胜敌者,附近以仁,来远以德,不与敌争,而敌自服也。}
善用人者为之下。
\zhushi{善用人自辅佐者,常为人执谦下也。}
是谓不争之德,
\zhushi{谓上为之下也。是乃不与人争之道德也。}
是谓用人之力,
\zhushi{能身为人下,是谓用人臣之力也。}
是谓配天古之极。
\zhushi{能行此者,德配天也。是乃古之极要道也。}


\section{玄用第六十九}

用兵有言:
\zhushi{陈用兵之道。老子疾时用兵,故托己设其义也。}
吾不敢为主而为客,
\zhushi{主,先也。不敢先举兵。客者,和而不倡。用兵当承天而后动。}
不敢进寸而退尺。
\zhushi{侵人境界,利人财宝,为进;闭门守城,为退。}
是谓行无行,
\zhushi{彼遂不止,为天下贼,虽行诛之,不成行列也。}
攘无臂,
\zhushi{虽欲大怒,若无臂可攘也。}
扔无敌,
\zhushi{虽欲仍引之,若无敌可仍也。}
执无兵。
\zhushi{虽欲执持之,若无兵刃可持用也。何者?伤彼之民罹罪于天,遭不道之君,愍忍丧之痛也。}
祸莫大于轻敌。
\zhushi{夫祸乱之害,莫大于欺轻敌家,侵取不休,轻战贪财也。}
轻敌,几丧吾宝。
\zhushi{几,近也。宝,身也。欺轻敌者,近丧身也。}
故抗兵相加,
\zhushi{两敌战也。}
哀者胜矣。
\zhushi{哀者慈仁,士卒不远于死。}


\section{知难第七十}

章吾言甚易知,甚易行。
\zhushi{老子言:吾所言省而易知,约而易行。}
天下莫能知,莫能行。
\zhushi{人恶柔弱,好刚强也。}
言有宗,事有君。
\zhushi{我所言有宗祖根本,事有君臣上下,世人不知者,非我之无德,心与我之反也。}
夫唯无知,是以不我知。
\zhushi{夫唯世人之无知者,是我德之暗,不见于外,穷微极妙,故无知也。}
知我者希,则我者贵。
\zhushi{希,少也。唯达道者乃能知我,故为贵也。}
是以圣人被褐怀玉。
\zhushi{被褐者薄外,怀玉者厚内,匿宝藏德,不以示人也。}


\section{知病第七十一}

知不知上,
\zhushi{知道言不知,是乃德之上。,}
不知知病。
\zhushi{不知道言知,是乃德之病。}
夫唯病病,是以不病。
\zhushi{夫唯能病苦众人有强知之病,是以不自病也。}
圣人不病,以其病病,是以不病。
\zhushi{圣人无此强知之病者,以其常苦众人有此病,以此非人,故不自病。夫圣人怀通达之知,托于不知者,欲使天下质朴忠正,各守纯性。小人不知道意,而妄行强知之事以自显著,内伤精神,减寿消年也。}


\section{爱己第七十二}

民不畏威,则大威至。
\zhushi{威,害也。人不畏小害则大害至。大害者,谓死亡也。畏之者当爱精神,承天顺地也。}
无狭其所居,
\zhushi{谓心居神,当宽柔,不当急狭也。}
无厌其所生,
\zhushi{人所以生者,以有精神。托空虚,喜清静,饮食不节,忽道念色,邪僻满腹,为伐本厌神也。}
夫唯不厌,是以不厌。
\zhushi{夫唯独不厌精神之人,洗心濯垢,恬泊无欲,则精神居之不厌也。}
是以圣人自知,不自见,
\zhushi{自知己之得失,不自显见德美于外,藏之于内。}
自爱不自贵。
\zhushi{自爱其身以保精气,不自贵高荣名于世。}
故去彼取此。
\zhushi{去彼自见、自贵,取此自知、自爱。}


\section{任为第七十三}

勇于敢则杀,
\zhushi{勇敢有为,则杀其身。}
勇于不敢则活。
\zhushi{勇于不敢有为,则活其身。}
此两者,
\zhushi{谓敢与不敢也。}
或利或害,
\zhushi{活身为利,杀身为害。}
天之所恶。
\zhushi{恶有为也。}
孰知其故?
\zhushi{谁能知天意之故而不犯?}
是以圣人犹难之。
\zhushi{言圣人之明德犹难于勇敢,况无圣人之德而欲行之乎?}
天之道,不争而善胜,
\zhushi{天不与人争贵贱,而人自畏之。}
不言而善应,
\zhushi{天不言,万物自动以应时。}
不召而自来,
\zhushi{天不呼召,万物皆负阴而向阳。}
繟然而善谋。
\zhushi{繟,宽也。天道虽宽博,善谋虑人事,修善行恶,各蒙其报也。}
天网恢恢,疏而不失。
\zhushi{天所网罗恢恢甚大,虽疏远,司察人善恶,无有所失。}


\section{制惑第七十四}

民不畏死,
\zhushi{治国者刑罚酷深,民不聊生,故不畏死也。治身者嗜欲伤神,贪财杀身,民不知畏之也。}
奈何以死惧之?
\zhushi{人君不宽刑罚,教民去情欲,奈何设刑法以死惧之?}
若使民常畏死,
\zhushi{当除己之所残克,教民去利欲也。}
而为奇者,吾得执而杀之。孰敢?
\zhushi{以道教化而民不从,反为奇巧,乃应王法执而杀之,谁敢有犯者?老子疾时王不先道德化之,而先刑罚也。}
常有司杀者。
\zhushi{司杀者,谓天居高临下,司察人过。天网恢恢,疏而不失也。}
夫代司杀者,是谓代大匠斫。
\zhushi{天道至明,司杀有常,犹春生夏长,秋收冬藏,斗杓运移,以节度行之。人君欲代杀之,是犹拙夫代大匠斫木,劳而无功也。}
夫代大匠斫者,希有不伤手矣。
\zhushi{人君行刑罚,犹拙夫代大匠斫,则方圆不得其理,还自伤。代天杀者,失纪纲,不得其纪纲还受其殃也。}


\section{贪损第七十五}

民之饥,以其上食税之多,是以饥。
\zhushi{人民所以饥寒者,以其君上税食下太多,民皆化上为贪,叛道违德,故饥。}
民之难治,以其上之有为,是以难治。
\zhushi{民之不可治者,以其君上多欲,好有为也。是以其民化上有为,情伪难治。}
民之轻死,以其上求生之厚,
\zhushi{人民所以侵犯死者,以其求生活之道太厚,贪利以自危。}
是以轻死。
\zhushi{以求生太厚之故,轻入死地也。}
夫唯无以生为者,是贤于贵生。
\zhushi{夫唯独无以生为务者,爵禄不干于意,财利不入于身,天子不得臣,诸侯不得使,则贤于贵生也。}


\section{戒强第七十六}

人之生也柔弱,
\zhushi{人生含和气,抱精神。故柔弱也。}
其死也坚强。
\zhushi{人死和气竭,精神亡,故坚强也。}
万物草木之生也柔脆,
\zhushi{和气存也。}
其死也枯槁。
\zhushi{和气去也。}
故坚强者死之徒,柔弱者生之徒。
\zhushi{以上二事观之,知坚强者死,柔弱者生也。}
是以兵强则不胜,
\zhushi{强大之兵轻战乐杀,毒流怨结,众弱为一强,故不胜。}
木强则共。
\zhushi{本强大则枝叶共生其上。}
强大处下,柔弱处上。
\zhushi{兴物造功,大木处下,小物处上。天道抑强扶弱,自然之效。}


\section{天道第七十七}

天之道,其犹张弓乎!
\zhushi{天道暗昧,举物类以为喻也。}
髙者抑之,下者举之,有馀者损之,不足者与之。
\zhushi{言张弓和调之,如是乃可用。夫抑髙举下,损强益弜,天之道也。}
天之道,损有馀而补不足。
\zhushi{天道损有馀而益谦,常以中和为上。}
人之道则不然,损不足以奉有馀。
\zhushi{人道则与天道反,世俗之人损贫以奉富,夺弱以益强也。}
孰能有馀以奉天下?唯有道者。
\zhushi{言谁能居有馀之位,自省爵禄以奉天下不足者乎?唯有道之君能行也。}
是以圣人为而不恃,
\zhushi{圣人为德施,不恃其报也。}
功成而不处,
\zhushi{功成事就,不处其位。}
其不欲见贤。
\zhushi{不欲使人知己之贤,匿功不居荣,畏天损有馀也。}


\section{任信第七十八}

天下柔弱莫过于水,
\zhushi{圆中则圆,方中则方,壅之则止,决之则行。}
而攻坚强者莫知能胜,
\zhushi{水能怀山襄陵,磨铁消铜,莫能胜水而成功也。}
其无以易之。
\zhushi{夫攻坚强者,无以易于水。}
弱之胜强,
\zhushi{水能灭火,阴能消阳。}
柔之胜刚,
\zhushi{舌柔齿刚,齿先舌亡。}
天下莫不知,
\zhushi{知柔弱者久长,刚强者折伤。}
莫能行。
\zhushi{耻谦卑,好强梁。}
故圣人云:
\zhushi{谓下事也。}
受国之垢,是谓社稷主;
\zhushi{人君能受国之垢浊者,若江海不逆小流,则能长保其社稷,为一国之君主也。}
受国不祥,是谓天下王。
\zhushi{人君能引过自与,代民受不祥之殃,则可以王天下。}
正言若反。
\zhushi{此乃正直之言,世人不知,以为反言。}


\section{任契第七十九}

和大怨,
\zhushi{杀人者死,伤人者刑,以相和报。}
必有馀怨,
\zhushi{任刑者失人情,必有馀怨及于良人也。}
安可以为善?
\zhushi{言一人,则先天心,安可以和怨为善?}
是以圣人执左契而不责于人。
\zhushi{古者圣人执左契,合符信也。无文书法律,刻契合符以为信也。但刻契为信,不责人以他事也。}
有德司契,
\zhushi{有德之君,司察契信而已。}
无德司彻。
\zhushi{无德之君,背其契信,司人所失。}
天道无亲,常与善人。
\zhushi{天道无有亲疏,唯与善人,则与司契同也。}


\section{独立第八十}

小国寡民,
\zhushi{圣人虽治大国,犹以为小,俭约不奢泰。民虽众,犹若寡少,不敢劳之也。}
使有什伯人之器而不用,
\zhushi{使民各有部曲什伯,贵贱不相犯也。器谓农人之器。而不用,不徵召夺民良时也。}
使民重死而不远徙。
\zhushi{君能为民兴利除害,各得其所,则民重死而贪生也。政令不烦则民安其业,故不远迁徙离其常处也。}
虽有舟舆,无所乘之;
\zhushi{清静无为,不作烦华,不好出入游娱也。}
虽有甲兵,无所陈之。
\zhushi{无怨恶于天下。}
使民复结绳而用之,
\zhushi{去文反质,信无欺也。}
甘其食,
\zhushi{甘其蔬食,不渔食百姓也。}
美其服,
\zhushi{美其恶衣,不贵五色。}
安其居,
\zhushi{安其茅茨,不好文饰之屋。}
乐其俗。
\zhushi{乐其质朴之俗,不转移也。}
邻国相望,鸡犬之声相闻,
\zhushi{相去近也。}
民至老不相往来。
\zhushi{其无情欲。}


\section{显质第八十一}

信言不美,
\zhushi{信者,如其实也。不美者,朴且质也。}
美言不信。
\zhushi{美言者,滋美之华辞。不信者,饰伪多空虚也。}
善者不辩,
\zhushi{善者,以道修身也。不彩文也。}
辩者不善。
\zhushi{辩者,谓巧言也。不善者,舌致患也。山有玉,掘其山;水有珠,浊其渊;辩口多言,亡其身。}
知者不博,
\zhushi{知者,谓知道之士。不博者,守一元也。}
博者不知。
\zhushi{博者,多见闻也。不知者,失要真也。}
圣人不积,
\zhushi{圣人积德不积财,有德以教愚,有财以与贫也。}
既以为人己愈有,
\zhushi{既以为人施设德化,己愈有德。}
既以与人己愈多。
\zhushi{既以财贿布施与人,而财益多,如日月之光,无有尽时。}
天之道,利而不害;
\zhushi{天生万物,爱育之,令长大,无所伤害也。}
圣人之道,为而不争。
\zhushi{圣人法天所施为,化成事就,不与下争功名,故能全其圣功也。}


\end{document}
